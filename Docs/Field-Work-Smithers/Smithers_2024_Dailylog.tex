\documentclass{article}\usepackage[]{graphicx}\usepackage[]{color}

\usepackage{alltt}
\usepackage{float}
\usepackage{graphicx}
\usepackage{tabularx}
\usepackage{siunitx}
\usepackage{amssymb} % for math symbols
\usepackage{amsmath} % for aligning equations
\usepackage{textcomp}
\usepackage{natbib}
\topmargin -1.5cm        
\oddsidemargin -0.04cm   
\evensidemargin -0.04cm
\textwidth 16.59cm
\textheight 21.94cm 
%\pagestyle{empty} %comment if want page numbers
\parskip 7.2pt
\renewcommand{\baselinestretch}{1.5}
\parindent 0pt
\usepackage{blindtext}
\usepackage[T1]{fontenc}
\usepackage[utf8]{inputenc}

\title{Smithers - 2024 - Dailylog}

\author{Deirdre Loughnan}

\begin{document}

\maketitle
\section*{Pre-fieldwork notes}

The 2024 field team will consist of: Deirdre Loughnan, Xiaomao Wang, Christophe Rouleau-Desrocher, and Britany Wu. The plan this year is to sample the seedlings and to start remeasuring the DBH of the adult trees for 1-3 plots to estimate how long it will take.  \\

We have a new contact at BC Parks, who asked to join us in the field for a day. I reached out on July 7 to coordinate this, but never received a reply.

\section*{July 15, 2024}

Left Vancouver at 12pm. We flipped the travel this year to give people more time at home, since the week prior we had the egret retreat and the week after we get back many people are heading to the field again. We stopped for lunch on the way out of town and drove until Hope, then onward to Juniper beach campground, just outside of Cache Creek. This campsite is located closer to the highway than Marble Canyon, but it was impossible to sleep with the noise from the two trains that sandwich the site. This is not nearly clear enough from the reviews! It was also oppressively hot (35 \textdegree C) and there are few non-fast food options for dinner.

\section*{July 16, 2024}

Drove the rest of the way to Smithers. I would not recommend leaving late the first day in the future if there are still only two drivers. It was pretty exhausting and having an afternoon to relax the second day is really helpful. It also gives us some time to go out and look for germinates before starting the sampling in full.

\section*{July 17, 2024}

First day of fieldwork! We had a late start, leaving the campsite around 8:45 and kept the day shortish, heading down the mountain at 4pm (probably arrived at the car around 5 or 5:30). We sampled sites 6, 12, and 13. At each site we sampled germinates, took soil moisture, recorded percent cover, and reflagged both the subplots and the corners. We had lofty plans to take the soil moisture for more plots (it is supposed to rain in two days), but we only finished the soil moisture and percent cover for plot 10. We decided the retractable measuring tape did not work well for measuring seedlings, so we need to stop in town and get a smaller ruler. 

\section*{July 18, 2024}

Got an earlier start today, leaving the campsite at 8:15. We stopped en route at Home Hardware to get a smaller ruler. Today we put in a much longer day and worked until 5:15 before hiking down. We hiked to the top of the transect and sampled: 2, 8, 17, 16, 15. This year germination seems really high with over 30 germinates in today alone! We again tried to get soil moisture and percent cover at sites before the rain, stopping at 14, 11, 9. Site 11 is by far our steepest site and we found the tags for plot A really far away from where they should have been. 

\section*{July 19, 2024}

There was a huge thunderstorm at 4:30/5 am this morning, it woke us all up so we are a bit tired today. We left the campsite around 8:15 and was back at the car by 5:30. Despite all this, we still put in a full day sampling seedlings at sites: 1, 5, 10, 7, 3, 4. 

\section*{July 20, 2024}

Got a good start to the day and had initially planned to finish all the seedling, but we seem to have lost the bag of toothpicks and ran out of ones we had in our pockets. We were able to sample 9 and start 14, but there are a lot of seedlings in 14. So instead we did DBH for 7 (45 min) and 3 (1 hour). Mao had the great idea of getting washable chalk to mark the trees. We used flagging to denote the edges of the plot (3 pieced of flagging along each side) and then divided into teams of two, with each team doing one side along the diagonal of the plot. In each team, one person took DBH and one person recorded the values. We wrote down all individuals when they were tagged, but few were. Measurements also included the trees that were flagged as the corners. In the evening we went to the dollarstore and grocery store looking for toothpicks. All they had were wooden ones or the ones with paper umbrellas. But we did find coloured dowels and crafting sticks at the dollarstore and bought those.

\section*{July 21, 2024}

Today we decided to split up as teams. As we hiked up we stopped to do soil moisture measurements at 1, 3, 4, 5, 7, and looked for the bag of toothpicks as we went. Christophe and I stopped at 11 to finish that plot, while Mao and Britany kept going up to 14 where they started to do DBH.  We tested the coloured dowels and found that they do run colour, but the permanent marker seems to stay, so we wrote both the colour and the shape on each one and finished the sampling. We missed searching for one seedling in plot 17 and decided that for fun we would hike up to the top of the trail and stope at 17 along the way. Christophe and Britany went all the way to the top, but Mao and I just relaxed at the treeline. The fire season seems to have started with a vengeance, there are fires near Cache Creek that might impact our drive home. To try and get ahead of them, we are planning to start the drive home tomorrow.

\section*{July 22, 2024}

We decided to take half a day to relax and have some fun before driving home. We drove up Mt. Hudson and hiked to Crater Lake. I also showed Christophe and Mao where Bob's cabin is and we saw a rare ptarmigan! We left Smithers around 1pm and drove to the West Lake campground. Unfortunately, when we got out of the car we could hear that our back tire on the passenger side had a puncture! Christophe was able to change the tire to the donut while I tried to find somewhere in town that had the right tires for us. It was 5:55 pm though and most places were closed. We also had another torrential downpour and thunderstorms!

\section*{July 23, 2024}

We were able to get an 11:30 pm appointment at Kal Tire! I called three placed in Prince George looking for tires and most places would have needed to order the right tires for us, which would have taken 2-4 days! We drove to a Tim Hortons near Kal Tire and worked for the morning. We hit the road at 1:30 pm. There were fires in Quesnel, Williams Lake, and Cache Creek, closing the highway through Cache Creek. We decided just to drive all the way home, taking the 5. We arrived at 11:30 in the city with everyone dropped off by 12:30. 

\section*{Things to discuss for next year:}

\begin{itemize}
\item Flagging subplot corners: can we think of a more permanent solution? like longer nails that will go deeper than the moss layer and be sturdier? If not, we must remember to bring new hair pins and tags for the ones that are getting rusty 
\item There are other types of data we might want to collect: using hobo loggers to get humidity and temperature, light, canopy cover, topography
\item Discuss longterm plans for site 9 (site E and F on the slope) and site 11 (the steepest site).
\item Bring multiple bags of toothpicks and keep some in the car incase they get lost
\end{itemize}

\end{document}